\section*{Введение}\label{sec:introduction}

\subsection*{Актуальность и релевантные работы}
Задача определения авторства возникает в разных областях на протяжении многих лет. Работы неизвестных или пожелавших остаться анонимными авторов существуют в литературе, живописи, музыке, и программирование не является исключением. Необходимость определить разработчика, стоящего за кодом, возникает при поиске авторов вредоносного программного обеспечения, решении вопросов интеллектуальной собственности.

Подходы к решению задачи определения авторства основываются на предположении, что каждый автор обладает уникальный стилем, точно так же как уникальным почерком или отпечатком пальца. Идея определения набора характеристик или метрик кода, которые отражают стиль его автора, интересует исследователей не менее трех десятилетий \cite{Halstead1972,Oman1989}. Статистический анализ стилистики кода, текстов или живописи называется стилометрией. Он основывается на подсчете метрик или факторов по исследуемому материалу и дальнейшем анализе полученных численных представлений.

Предыдущие исследования в области определения авторства кода были мотивированы проверкой авторства \cite{Elenbogen2008,Kothari2007}, поиском плагиата \cite{Ottenstein1976,Liu2006}, определением авторов вредоносных программ \cite{Layton2010,Caliskan2015}. Для использования автоматических методов во всех перечисленных задачах требуется хорошее тестирование. Для этого необходим набор данных, на котором оно будет производиться. В существующих работах в качестве датасетов применялись студенческие домашние задания \cite{Frantzeskou2007,Elenbogen2008,Krsul1997}, примеры кода из книг \cite{Oman1989,Frantzeskou2006}, задачи соревнований по программированию \cite{Rosenblum2011,Caliskan2015,Alsulami2017,Simko2018} и проекты с открытым исходным, написанные одним автором \cite{Shevertalov2009,Kothari2007,Yang2017,Zhang2017}. У всех перечисленных типов датасетов есть свои недостатки: они отличаются от промышленного кода и у них есть ограничение на количество примеров, доступных для каждого автора.

В предыдущих работах не изучалось масштабирование решений на случай, когда для каждого автора доступны не десятки или сотни примеров, а десятки тысяч. Это связано с ограничениями имеющихся датасетов: такие объемы доступны только в крупных промышленных проектах, а в них одновременно работает большое количество программистов и сложно разделить код между ними, чтобы собрать примеры, для которых автор известен. Помимо этого, лучшие модели на данный момент для определения авторства по коду на языках C++ и Java используют факторы, специфичные для конкретных языков, что усложняет использование предложенных методов для других языков.

\subsection*{Цель и задачи}
Данная работа ставит целью создание универсального относительно языка решения, которое улучшает или повторяет существующие результаты в определении авторства для C++, Java и Python, проверка его масштабируемости на разные объемы данных, доступных для каждого автора, сравнение с имеющимися исследованиями. Для этого ставятся следующие задачи:
\begin{itemize}
    \item Научиться собирать данные для обучения и тестирования моделей, определяющих авторство кода, из проектов с произвольным количеством авторов.
    \item Создать набор датасетов для анализа применимости моделей в разных условиях: при разном количестве данных, несбалансированном количестве примеров, доступном для каждого автора, разделяя примеры в обучающей и тестовой выборке по времени.
    \item Создать модель, позволяющую работать одинаковым образом с кодом на различных языках и адаптировать её для работы с произвольным объемом доступных данных.
    \item Протестировать модель на собранных датасетах и сравнить с существующими решениями, используя данные, на которых они тестировались.
\end{itemize}

\subsection*{Достигнутые результаты}
В рамках данной работы был реализован инструмент для сбора данных из проектов с произвольным числом автором, которые разрабатывались с использованием системы контроля версия (VCS, version control system) git. Для этого история проекта разбивается на отдельные изменения, для каждого из которых известен автор. Эти изменения в свою очередь разбиваются на изменения методов или классов, которые и составляют итоговый датасет. Такой подход снимает ограничения, которые были у предыдущих наборов данных: есть возможность использовать промышленный код, нет ограничения в выборе проектов и количестве доступных примеров.

При помощи инструмента были собраны данные обо всех изменениях методов в IntelliJ IDEA\footnote{\url{https://github.com/jetbrains/intellij-community}}, втором по размеру проекте на GitHub с открытым исходным кодом на языке Java. Это 700 тысяч созданных методов и 2 миллиона изменений, принадлежащих 500 разработчикам. Из них было составлено 5 датасетов, предназначенных для тестирования моделей в разных условиях (подробнее в главе 3).

Было реализовано две модели: для работы в условиях небольшого количества примеров (от нескольких до сотен), доступных для каждого разработчика и в условиях большого (десятки и сотни тысяч). Обе модели основываются на технике представления кода при помощи контекстов в абстрактном синтаксическом дереве (AST, abstract syntax tree) \cite{Alon2018}. Использование AST для определения авторства по коду уже позволяло улучшать результаты в нескольких недавних работах \cite{Caliskan2015,Alsulami2017}.

Полученные модели были протестированы на наборах данных для C++, Java и Python, результаты тестирования на которых приводили исследователи в недавних работах. Полученная точность в случае Python и Java оказалась лучше, чем у предыдущих решений и повторила предыдущие результаты в случае C++ (подробнее в главе 6). Для датасетов, собранных на основе IntelliJ IDEA, результаты не сравниваются с предыдущими решениями из-за отсутствия их реализации в открытом доступе или того, что имеющиеся реализации не поддерживают Java.

\subsection*{Структура работы}
В главе 1 представлен обзор предыдущих исследований в области, история развития подходов к определению авторства с точки зрения данных, факторов и методов машинного обучения. 

В главе 2 представлена предложенная техника сбора данных из проектов с произвольным числом авторов, реализованный инструмент для сбора данных и собранные датасеты.

В главе 3 приводятся технические детали и архитектура созданных моделей.

В главе 4 представлены численные результаты проделанной работы, результаты проведенных экспериментов и приведено исследование поведения моделей в разных условиях.

В последней главе анализируется проделанная работа и приводятся возможные перспективы дальнейшей деятельности.